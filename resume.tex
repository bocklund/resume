% resume.tex

\documentclass[10pt,letterpaper]{article}
\usepackage[letterpaper,margin=0.55in, top = 0.3in, bottom = 0.3in]{geometry}
\usepackage[utf8]{inputenc}
\usepackage{mdwlist}
\usepackage[T1]{fontenc}
\usepackage{textcomp}
\usepackage{amsmath}
\usepackage{tgpagella}
\usepackage[colorlinks=true, urlcolor=blue]{hyperref}
\pagestyle{empty}
\setlength{\tabcolsep}{0em}

% indent section style, used for sections that aren't already in lists
% that need indentation to the level of all text in the document
\newenvironment{indentsection}[1]%
{\begin{list}{}%
  {\setlength{\leftmargin}{#1}}%
  \item[]%
}
{\end{list}}

% opposite of above; bump a section back toward the left margin
\newenvironment{unindentsection}[1]%
{\begin{list}{}%
  {\setlength{\leftmargin}{-0.5#1}}%
  \item[]%
}
{\end{list}}

% format two pieces of text, one left aligned and one right aligned
\newcommand{\headerrow}[2]
%{\begin{tabular*}{\linewidth}{l@{\extracolsep{\fill}}r}
{\begin{tabular*}{\linewidth}{l@{\extracolsep{\fill}}r}
  #1 &
  #2 \\
\end{tabular*}}

% make "C++" look pretty when used in text by touching up the plus signs
\newcommand{\CPP}
{C\nolinebreak[4]\hspace{-.05em}\raisebox{.22ex}{\footnotesize\bf ++}}

% and the actual content starts here
\begin{document}
\begin{center}
{\LARGE \textbf{Brandon Bocklund}}

%ADDRESS LINE 1\ \ \textbullet
%\ \ ADDRESS LINE 2\ \ \textbullet
%\ \ CITY, ST #ZIP#
%\\


(269) 589-8602\ \ \textbullet
\ \ \href{mailto:bocklund@psu.edu}{bocklund@psu.edu}
\end{center}

\hrule
\vspace{-0.6em}
\subsection*{Research Experience}

\renewcommand\labelitemiii{$\circ$}
\begin{itemize}
  \parskip=0.1em

  \item
  \headerrow
    {\textbf{Phases Research Lab, Pennsylvania State University}}
    {\textbf{University Park, PA}}
  \\
  \headerrow
    {\emph{NASA Space Technology Research Fellow (Advisor: Zi-Kui Liu)}}
    {\emph{2016 -- Present}}
  \begin{itemize*}
    \item Developed ESPEI, a user tool for multicomponent CALPHAD modeling and uncertainty quantification
    \item Developed DFTTK, a high-throughput framework for \emph{ab-initio} quasiharmonic phonon calculations with VASP
    \item Research projects:
    \begin{itemize}
      \item Joining stainless steel to Ti-6Al-4V in functionally graded, additively manufactured alloys
      \item Applying ESPEI to quantify uncertainty in a new CALPHAD description of Cu-Mg
      \item Deriving a thermodynamic model to predict the effect of oxygen impurities on glass formability of BMGs
    \end{itemize}
  \end{itemize*}

  \item
  \headerrow
    {\textbf{Solid State Ionics Laboratory, Michigan State University}}
    {\textbf{East Lansing, MI}}
  \\
  \headerrow
    {\emph{Undergraduate Research Assistant (Advisor: Jason D. Nicholas)}}
    {\emph{2015 -- 2016}}
  \begin{itemize*}
    \item Fabricated and improved the performance of solid oxide fuel cells
    \item Characterized fuel cells with EIS, XRD, and SEM
    \item Developed Rp Plotter, a GUI-based Python application for data analysis and visualization
    \item Participated in a 10 week professional development course
  \end{itemize*}

  \item
  \headerrow
    {\textbf{Composite Materials \& Structures Center, Michigan State University}}
    {\textbf{East Lansing, MI}}
  \\
  \headerrow
    {\emph{Undergraduate Research Assistant (Advisor: Lawrence T. Drzal)}}
    {\emph{2014 -- 2015}}
  \begin{itemize*}
    \item Designed a graphene nanoplatlet-based capacitive deionization cell
    \item Characterized graphene nanoplatelet papers using scanning electron microscopy
    \item Used Solidworks to create a 3D printed model for the deionization cell apparatus
    \item Participated in a 10 week professional development course
  \end{itemize*}

\end{itemize}

\hrule
\vspace{-0.6em}
\subsection*{Teaching Experience}

\renewcommand\labelitemiii{$\circ$}
\begin{itemize}
    \parskip=0.1em

    \item
    \headerrow
    {\textbf{Department of Materials Science and Engineering, Penn State University}}
    {\textbf{State College, PA}}
    \\
    \headerrow
    {\emph{Teaching Assistant}}
    {\emph{2017}}
    \begin{itemize*}
        \item MatSE 404/BME 444: Surfaces and the Biological Response to Materials
            \begin{itemize}
                \item Developed and graded problems for homework and exams
            \end{itemize}
        \item MatSE 462: General Properties Laboratory in Materials
            \begin{itemize}
                \item Independently taught and graded assignments for two lab sections of 5 students
                \item Instructed students on using techniques for characterizing mechanical, electrical and optical properties
            \end{itemize}
    \end{itemize*}

    \item
    \headerrow
    {\textbf{College of Engineering, Michigan State University}}
    {\textbf{East Lansing, MI}}
    \\
    \headerrow
    {\emph{Undergraduate Lab Mentor}}
    {\emph{2015 -- 2016}}
\end{itemize}


\hrule
\vspace{-0.6em}
\subsection*{Education}

\begin{itemize}
  \parskip=0.1em

  \item
  \headerrow
    {\textbf{Pennsylvania State University}}
    {\textbf{University Park, PA}}
  \\
  \headerrow
    {\emph{Ph.D. Materials Science and Engineering; Graduate Minor, Computational Materials}}
    {\emph{2016 -- Present}}
  \begin{itemize*}
    \item 3.7 GPA
    \item NASA Space Technology Research Fellow (2018 -- Present)
    \item Honorable Mention, National Science Foundation Graduate Research Fellowship Program (2018)
    \item NSF Research Trainee in the CoMET Program (\href{http://dftcomet.psu.edu}{dftcomet.psu.edu}) (2016 -- 2018)
    \item Helen R. and Van H. Leichliter Graduate Fellowship recipient (2016)
  \end{itemize*}


  \item
  \headerrow
    {\textbf{Michigan State University}}
    {\textbf{East Lansing, MI}}
  \\
  \headerrow
    {\emph{B.S. Materials Science and Engineering}}
    {\emph{2012 -- 2016}}
  \begin{itemize*}
    \item 3.56 GPA
    \item Dean's List, 5 semesters
    \item MSU College of Engineering Endowed Opportunity Fund scholarship recipient (2015 -- 2016)
  \end{itemize*}

\end{itemize}

\hrule
\vspace{-0.6em}
\subsection*{Technical Skills}

\begin{indentsection}{\parindent}
\hyphenpenalty=1000
\begin{description*}
  \item [Software Developed:]
  ESPEI (\href{https://espei.org}{espei.org}), pycalphad (\href{https://pycalphad.org}{pycalphad.org}), DFTTK (\href{https://github.com/PhasesResearchLab/DFTTK}{github.com/phasesresearchlab/dfttk})
  \item[Computational Tools and Software:]
  Python, VASP, pycalphad, Thermo-Calc, MongoDB
\end{description*}
\end{indentsection}
\end{document}
