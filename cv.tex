% resume.tex

\documentclass[10pt,letterpaper]{article}
\usepackage[letterpaper,margin=0.55in, top = 0.3in, bottom = 0.3in]{geometry}
\usepackage[utf8]{inputenc}
\usepackage{ifthen}
\usepackage{mdwlist}
\usepackage[T1]{fontenc}
\usepackage{textcomp}
\usepackage{amsmath}
\usepackage{etaremune}  % enumerate, backwards
\usepackage{tgpagella}
\usepackage[colorlinks=true, urlcolor=blue]{hyperref}
\pagestyle{empty}
\setlength{\tabcolsep}{0em}

\usepackage{fontspec}
\defaultfontfeatures{Path={/Users/brandon/Adobe-Utopia-fonts-named/}}
\setmainfont[
  UprightFont=*-Regular,
  ItalicFont=*-Italic,
  BoldFont=*-Semibold,
  BoldItalicFont=*-SemiboldIt,
]{UtopiaStd}


% indent section style, used for sections that aren't already in lists
% that need indentation to the level of all text in the document
\newenvironment{indentsection}[1]%
{\begin{list}{}%
  {\setlength{\leftmargin}{#1}}%
  \item[]%
}
{\end{list}}

% opposite of above; bump a section back toward the left margin
\newenvironment{unindentsection}[1]%
{\begin{list}{}%
  {\setlength{\leftmargin}{-0.5#1}}%
  \item[]%
}
{\end{list}}

% format two pieces of text, one left aligned and one right aligned
\newcommand{\headerrow}[2]
%{\begin{tabular*}{\linewidth}{l@{\extracolsep{\fill}}r}
{\begin{tabular*}{\linewidth}{l@{\extracolsep{\fill}}r}
  #1 &
  #2 \\
\end{tabular*}}

% make "C++" look pretty when used in text by touching up the plus signs
\newcommand{\CPP}
{C\nolinebreak[4]\hspace{-.05em}\raisebox{.22ex}{\footnotesize\bf ++}}

%%%%%%%%%%%%%%%%%%%%%%%%%%%
%%% CONFIGURATION FLAGS %%%
%%%%%%%%%%%%%%%%%%%%%%%%%%%

% Boolean flags for which type of conference presentations are displayed
\newboolean{USE_CONFERENCES_SECTION}
\newboolean{USE_CONFERENCE_POSTER}
\newboolean{USE_CONFERENCE_PRESENTATION}
\newboolean{USE_CONFERENCE_INVITED_PRESENTATION}

% Should the conference section be used at all?
\setboolean{USE_CONFERENCES_SECTION}{true}
% Should posters be printed?
\setboolean{USE_CONFERENCE_POSTER}{false}
% Should regular talks be printed
\setboolean{USE_CONFERENCE_PRESENTATION}{true}
% Should invited talks be printed?
\setboolean{USE_CONFERENCE_INVITED_PRESENTATION}{true}

% Boolean flag for whether technical skills should be shown
\newboolean{USE_TECHNICAL_SKILLS}
\setboolean{USE_TECHNICAL_SKILLS}{false}

% Boolean flag for whether publications are enumerated
\newboolean{ENUMERATE_PUBLICATIONS}
\setboolean{ENUMERATE_PUBLICATIONS}{true}

% Boolean flag for whether presentations are enumerated
\newboolean{ENUMERATE_PRESENTATIONS}
\setboolean{ENUMERATE_PRESENTATIONS}{true}

% Boolean flag for whether to show TAing details
\newboolean{SHOW_TA_DETAILS}
\setboolean{SHOW_TA_DETAILS}{false}



% and the actual content starts here
\begin{document}
\begin{center}
{\LARGE \textbf{Brandon Bocklund}}

%ADDRESS LINE 1\ \ \textbullet
%\ \ ADDRESS LINE 2\ \ \textbullet
%\ \ CITY, ST #ZIP#
%\\


(269) 589-8602\ \ \textbullet
\ \ \href{mailto:bocklund@psu.edu}{bocklund@psu.edu}
\end{center}

\hrule
\vspace{-0.6em}
\subsection*{Research Experience}

\renewcommand\labelitemiii{$\circ$}
\begin{itemize}
  \parskip=0.1em
  \hyphenpenalty=1000

  \item
  \headerrow
    {\textbf{Phases Research Lab, Pennsylvania State University}}
    {\textbf{University Park, PA}}
  \\
  \headerrow
    {\emph{NASA Space Technology Research Fellow (Advisor: Zi-Kui Liu)}}
    {\emph{2016 -- Present}}
  \begin{itemize*}
    \item Developed uncertainty quantification methods for CALPHAD modeling
    through development of pycalphad and ESPEI, open research and education
    software for computational thermodynamics
    \item Established ICME approaches for designing functionally graded, additively
    manufactured materials
    \item Designed methods for high-throughput first-principles DFT calculations
    for metals and alloys
    \item Mentored undergraduate students in the Women In Science and Engineering
      Research (WISER) program
  \end{itemize*}

  \item
  \headerrow
    {\textbf{NASA Jet Propulsion Lab}}
    {\textbf{La Ca\~nada Flintridge, CA}}
  \\
  \headerrow
    {\emph{Graduate Research Intern (Mentors: Richard Otis, Peter Dillon)}}
    {\emph{05/2017 -- 08/2017}}
  \begin{itemize*}
    \item Used computational thermodynamics to develop bulk metallic glass alloy composition specifications
    \item Developed a model for oxygen tolerance in bulk metallic glasses
  \end{itemize*}

  \item
  \headerrow
    {\textbf{Solid State Ionics Laboratory, Michigan State University}}
    {\textbf{East Lansing, MI}}
  \\
  \headerrow
    {\emph{Undergraduate Research Assistant (Advisor: Jason D. Nicholas)}}
    {\emph{2015 -- 2016}}
  \begin{itemize*}
    \item Fabricated and improved the performance of solid oxide fuel cells
    \item Characterized fuel cells with EIS, XRD, and SEM
%    \item Developed Rp Plotter, a GUI-based Python application for data analysis and visualization
%    \item Participated in a 10 week professional development course
  \end{itemize*}

  \item
  \headerrow
    {\textbf{Composite Materials \& Structures Center, Michigan State University}}
    {\textbf{East Lansing, MI}}
  \\
  \headerrow
    {\emph{Undergraduate Research Assistant (Advisor: Lawrence T. Drzal)}}
    {\emph{2014 -- 2015}}
  \begin{itemize*}
    \item Designed a graphene nanoplatlet-based capacitive deionization cell
    \item Characterized graphene nanoplatelet papers using scanning electron microscopy
%    \item Used Solidworks to create a 3D printed model for the deionization cell apparatus
%    \item Participated in a 10 week professional development course
  \end{itemize*}

\end{itemize}

\hrule
\vspace{-0.6em}
\subsection*{Teaching Experience}

\renewcommand\labelitemiii{$\circ$}
\begin{itemize}
    \parskip=0.1em

    \item
    \headerrow
    {\textbf{Department of Materials Science and Engineering, Pennsylvania State University}}
    {\textbf{University Park, PA}}
    \\
    \headerrow
    {\emph{Teaching Assistant}}
    {\emph{2016 -- Present}}
    {} % right hand side text
    \begin{itemize*}
        \item \emph{(Spring 2020)} MatSE 410: Phase Relations in Materials Systems
\ifthenelse{\boolean{SHOW_TA_DETAILS}}{
            \begin{itemize}
                \item Developed problem sets for homework
                \item Graded homework and exams
            \end{itemize}
}{}
        \item \emph{(Spring 2018)} MatSE 404/BME 444: Surfaces and the Biological Response to Materials
\ifthenelse{\boolean{SHOW_TA_DETAILS}}{
            \begin{itemize}
                \item Developed and graded problems for homework and exams
            \end{itemize}
}{}
        \item \emph{(Spring 2017)} MatSE 462: General Properties Laboratory in Materials
\ifthenelse{\boolean{SHOW_TA_DETAILS}}{
            \begin{itemize}
                \item Independently taught and graded assignments for two lab sections of 10 students
                \item Instructed students on using techniques for characterizing mechanical, electrical and optical properties
            \end{itemize}
}{}
    \end{itemize*}

    \item
    \headerrow
    {\textbf{College of Engineering, Michigan State University}}
    {\textbf{East Lansing, MI}}
    \\
    \headerrow
    {\emph{Undergraduate Lab Mentor}}
    {\emph{2015 -- 2016}}
\ifthenelse{\boolean{SHOW_TA_DETAILS}}{
    \begin{itemize*}
        \item Mentored 3 classes, interacting with over 250 students
        \item Responsible for grading assignments and quizzes, promoting learning, and proctoring exams
        \begin{itemize}
            \item EGR 100: Introduction to Engineering Design
            \item EGR 102: Introduction to Engineering Modeling
            \item EGR 291: Spatial Visualization
        \end{itemize}
    \end{itemize*}
}{
    \begin{itemize*}
        \item \emph{(Spring 2016)} EGR 102: Introduction to Engineering Modeling
        \item \emph{(Fall 2015)} EGR 100: Introduction to Engineering Design
        \item \emph{(Fall 2015)} EGR 291: Spatial Visualization
    \end{itemize*}
}
\end{itemize}


\hrule
\vspace{-0.6em}
\subsection*{Education}

\begin{itemize}
  \parskip=0.1em

  \item
  \headerrow
    {\textbf{Pennsylvania State University}}
    {\textbf{University Park, PA}}
  \\
  \headerrow
    {\emph{Ph.D., Materials Science and Engineering; Graduate Minor, Computational Materials}}
    {\emph{2016 -- Present}}
  \begin{itemize*}
    \item 3.74 GPA
    \item NASA Space Technology Research Fellow (2018 -- Present)
    \item NSF Research Trainee in the CoMET Program (\href{http://dftcomet.psu.edu}{dftcomet.psu.edu}) (2016 -- 2018)
  \end{itemize*}


  \item
  \headerrow
    {\textbf{Michigan State University}}
    {\textbf{East Lansing, MI}}
  \\
  \headerrow
    {\emph{B.S. Materials Science and Engineering}}
    {\emph{2012 -- 2016}}
  \begin{itemize*}
    \item 3.56 GPA
    \item Dean's List, 5 semesters
  \end{itemize*}

\end{itemize}

\hrule
\vspace{-0.6em}
\subsection*{Awards and Honors}

\begin{itemize}
  \parskip=0.1em

  \item
  \headerrow
       {Runner Up, NASA Software of the Year (SoY) award - \emph{pycalphad}}
       {\emph{2019}}
  \item
  \headerrow
       {Honorable Mention, National Science Foundation Graduate Research Fellowship Program}
       {\emph{2018}}
  \item
  \headerrow
    {Outstanding Contribution in Reviewing - \emph{CALPHAD Journal}}
    {\emph{2017}}
  \item
  \headerrow
    {Helen R. and Van H. Leichliter Graduate Fellowship recipient} {\emph{2016}}
  \item
  \headerrow
    {MSU College of Engineering Endowed Opportunity Fund scholarship recipient} {\emph{2015}}

\end{itemize}


\hrule
\vspace{-0.6em}
\subsection*{Publications}
% TODO: Try to use bibliographic files here.
% TODO: Break out into a separate file
% TODO: Have some kind of highlighted flagging system where a short list could be generated with only key highlighted references.

\ifthenelse{\boolean{ENUMERATE_PUBLICATIONS}}
{\begin{etaremune}}
{\begin{itemize*}}

% Formatting:
% \item <Authors>
% <Title>
% <Publication>
% <Link>

%%%%%%%%%%%% Submitted (no citation information) %%%%%%%%%%%%
% Scheil FGMs
\item \textbf{B. Bocklund}, L.D. Bobbio, R.A. Otis, A.M. Beese, Z.-K. Liu,
Scheil-Gulliver simulations for the design of functionally graded alloys by additive manufacturing using pycalphad,
\emph{Submitted}
%\href{url}{doi:text}.

% Setareh automatic weighting
\item S. Zomorodpoosh, \textbf{B. Bocklund}, A. Obaied, R. Otis, Z.-K. Liu, I. Roslyakova,
Statistical approach for automated weighting of datasets: Application to heat capacity data,
\emph{Submitted}

% Abdulmonem pure Cr reassessment
\item A. Obaied, \textbf{B. Bocklund}, S. Zomorodpoosh, L. Zhang, R. Otis, Z.-K. Liu, I. Roslyakova,
  Thermodynamic re-assessment of pure chromium using modified segmented regression model,
  \textbf{CALPHAD} \emph{Accepted} (2019).


%%%%%%%%%%%% 2020 %%%%%%%%%%%%
% FGMs: Sigma in SS304L compared to SS420
\item L.D. Bobbio, \textbf{B. Bocklund}, A. Reichardt, R.A. Otis, J.P. Borgonia, R.P. Dillon, A.A. Shapiro, B.W. McEnerney, P. Hosemann, Z.-K. Liu, A.M. Beese,
Analysis of formation and growth of the $ \sigma $ phase in additively manufactured functionally graded materials,
\textbf{Journal of Alloys and Compounds} 814 (2020) 151729.
\href{https://doi.org/10.1016/j.jallcom.2019.151729}{doi:10.1016/j.jallcom.2019.151729}.

%%%%%%%%%%%% 2019 %%%%%%%%%%%%
% ESPEI
\item \textbf{B. Bocklund}, R.A. Otis, A. Egorov, A. Obaied, I. Roslyakova, Z.-K. Liu,
  ESPEI for efficient thermodynamic database development, modification, and uncertainty quantification: application to Cu-Mg,
  \textbf{MRS Communications} 9(2) (2019) 618-627.
  \href{https://doi.org/10.1557/mrc.2019.59}{doi:10.1557/mrc.2019.59}.

% Noah phase diagram UQ
\item N.H. Paulson, \textbf{B. Bocklund}, R.A. Otis, Z.-K. Liu, S. Marius,
  Quantified Uncertainty in Thermodynamic Modeling for Materials Design.
  \textbf{Acta Materialia} 174 (2019) 9-15.
  \href{https://doi.org/10.1016/j.actamat.2019.05.017}{doi:10.1016/j.actamat.2019.05.017}.

%%%%%%%%%%%% 2018 %%%%%%%%%%%%
% Seebeck
\item Y. Wang, Y.-J. Hu, \textbf{B. Bocklund}, S.-L. Shang, B.-C. Zhou, Z.-K. Liu, L.-Q. Chen,
  First-principles thermodynamic theory of Seebeck coefficients,
  \textbf{Physical Review B} 98 (2018) 224101.
  \href{https://doi.org/10.1103/PhysRevB.98.224101}{doi:10.1103/PhysRevB.98.224101}.

% FGMs: Ti-6Al-4V to 304L
\item L.D. Bobbio, \textbf{B. Bocklund}, R.A. Otis, J.P. Borgonia, R.P. Dillon, A.A. Shapiro, B. McEnerney, Z.-K. Liu, A.M. Beese,
  Characterization of a functionally graded material of Ti-6Al-4V to 304L stainless steel with an intermediate V section. \textbf{Journal of Alloys and Compounds} 742 (2018) 1031-1036.
  \href{https://doi.org/10.1016/j.jallcom.2018.01.156}{doi: 10.1016/j.jallcom.2018.01.156}

% FGMs: V to Invar36
\item L.D. Bobbio, \textbf{B. Bocklund}, R.A. Otis, J.P. Borgonia, R.P. Dillon, A.A. Shapiro, B. McEnerney, Z.-K. Liu, A.M. Beese,
  Experimental analysis and thermodynamic calculations of an additively manufactured functionally graded material of V to Invar 36,
  \textbf{Journal of Materials Research} 33 (2018) 1642–1649.
  \href{https://doi.org/10.1016/10.1557/jmr.2018.92}{doi:10.1557/jmr.2018.92}.

%%%%%%%%%%%% 2017 %%%%%%%%%%%%
% atomate
\item K. Mathew, J.H. Montoya, A. Faghaninia, S. Dwarakanath, M. Aykol,
  H. Tang, I. Chu, T. Smidt, \textbf{B. Bocklund}, M. Horton, J. Dagdelen, B. Wood, Z.-K. Liu, J. Neaton, S.P. Ong, K. Persson, A. Jain,
 Atomate: A high-level interface to generate, execute,
  and analyze computational materials science workflows. \textbf{Computational Materials Science} 139, 140–152 (2017).
  \href{https://doi.org/10.1016/j.commatsci.2017.07.030}{doi: 10.1016/j.commatsci.2017.07.030}

\ifthenelse{\boolean{ENUMERATE_PUBLICATIONS}}
{\end{etaremune}}
{\end{itemize*}}


\ifthenelse{\boolean{USE_CONFERENCES_SECTION}}{
\hrule
\vspace{-0.6em}
\subsection*{Presentations}

%!TEX root = cv.tex
% Assumes ifthen package
% Defines 3 commands that correspond to boolean values
%   1. \confposter        :  USE_CONFERENCE_POSTER                :  conference posters
%   2. \confpresentation  :  USE_CONFERENCE_PRESENTATION          :  conference talks
%   3. \confinvited       :  USE_CONFERENCE_INVITED_PRESENTATION  :  conference invited talks
% Each command will create a new item (in an itemized environment) and spit out the text if the corresponding boolean flag is enabled.
% Note: ``confinvited'' is also being used for workshops since that's what I equate to the status of them

\newcommand{\confposter}[1]{\ifthenelse{\boolean{USE_CONFERENCE_POSTER}}{\item #1}{}}
\newcommand{\confpresentation}[1]{\ifthenelse{\boolean{USE_CONFERENCE_PRESENTATION}}{\item #1}{}}
\newcommand{\confinvited}[1]{\ifthenelse{\boolean{USE_CONFERENCE_INVITED_PRESENTATION}}{\item #1}{}}

% TODO: try to use bibliographic files here.
% Formatting:
% <Authors>
% <(Year, Month)>
% <Title>
% <Name of conference>,
% <Location>

% Reverse number publications and presentations
\ifthenelse{\boolean{ENUMERATE_PRESENTATIONS}}
{\begin{etaremune}}
{\begin{itemize*}}

% All presentations here should be ones that /I/ presented.
% TODO: Submitted/accepted?

% Calphad 2024
\confpresentation {
\textbf{B. Bocklund*}, A. Perron, K. Bertsch
(2024, May).
Implementation of an extensible property modeling framework in ESPEI.
Calphad LI,
Stockholm, Sweden.
}


% CMI Winter Meeting 2024
\confposter {
\textbf{B. Bocklund*}, N. Ury, A. Perron
(2024, April).
Progress in thermodynamic modeling of molten salt separation processes for NdCl3.
CMI Hub Meeting,
Golden, CO.
}

% CALPHAD 2023, Workshop
\confinvited {
\textbf{B. Bocklund*}, R.A. Otis*
(2023, June).
\emph{Software Workshop:} PyCalphad and ESPEI.
CALPHAD 50,
Boston, MA.
}

% TMS 2023
\confinvited {
\textbf{B. Bocklund*}, A. Perron
(2023, March).
Rapidly generating Calphad databases with high-throughput first-principles calculations.
TMS 2023 Annual Meeting,
San Diego, CA.
}

% STOHT Meeting 2022
\confinvited {
\textbf{B. Bocklund*}, R.A. Otis*
(2022, September).
\emph{Software Workshop:} PyCalphad and ESPEI.
Structure and Thermodynamics of Oxides/carbides/nitrides/borides at High Temperatures (STOHT),
Tempe, AZ.
}

% Calphad 2022
\confpresentation {
\textbf{B. Bocklund*}, R.A. Otis, A. Perron, Z.-K. Liu
(2022, May).
A general approach for computing the residuals between CALPHAD models and phase diagram data.
Calphad XLIX,
Stockholm, Sweden.
}


% TMS 2022
\confposter {
\textbf{B. Bocklund*}, R.A. Otis, Z.-K. Liu
(2022, March).
Uncertainty Quantification and
    Propagation in CALPHAD Modeling.
TMS 2022 Annual Meeting,
Anaheim, CA.
}

% TMS 2020
\confpresentation {
\textbf{B. Bocklund*}, R.A. Otis, Z.-K. Liu
(2020, February).
Uncertainty quantification and propagation in ICME enabled by ESPEI.
TMS 2020 Annual Meeting,
San Diego, CA.
}

% MS&T 2019
\confinvited {
\textbf{B. Bocklund*}, R.A. Otis, Z.-K. Liu
(2019, October) \emph{Invited}.
Automated CALPHAD modeling and uncertanty quantification of a ternary system using ESPEI.
Materials Science and Technology 2019,
Portland, OR.
}

% CALPHAD 2019, Workshop
\confinvited {
\textbf{B. Bocklund}, R.A. Otis*, Z.-K. Liu
(2019, June).
\emph{Software Workshop:} PyCalphad and ESPEI.
CALPHAD XLVIII,
Singapore.
}

% CALPHAD 2019, Noah's talk that I gave
\confpresentation {
N.H. Paulson, \textbf{B. Bocklund*}, R.A. Otis, Z.-K. Liu, M. Stan
(2019, June).
Quantified Uncertainty in CALPHAD for Materials Design.
CALPHAD XLVIII,
Singapore.
}

% CALPHAD 2019, Matt's poster
\confposter {
M. Feurer, \textbf{B. Bocklund*}, S. Shang, A. Beese, Z.-K. Liu
(2019, June).
High-Throughput Modeling of Cr-Fe-Ni Sigma Phase.
CALPHAD XLVIII,
Singapore.
}

% CALPHAD 2019, my poster
\confposter {
\textbf{B. Bocklund*}, R.A. Otis, Z.-K. Liu
(2019, June).
Automated CALPHAD modeling and uncertanty quantification of a ternary system using ESPEI.
CALPHAD XLVIII,
Singapore.
}

% TMS 2019
\confpresentation{
\textbf{B. Bocklund*}, L.D. Bobbio, R.A. Otis, S. Shang, A.M. Beese, Z.-K. Liu
(2019, March).
Impact of Uncertainty Quantification in Automated Calphad Modeling on the design of Additively Manufactured, Functionally-graded Alloys.
TMS 2019 Annual Meeting,
Phoenix, AZ.
}

% MS&T 2018
\confpresentation {
\textbf{B. Bocklund*}, R.A. Otis, Z.-K. Liu
(2018, October).
Computational Tools for the Automated Development of a Cr-Fe-Ni-Ti-V CALHPAD Database.
Materials Science and Technology 2018,
Columbus, OH.
}

% CALPHAD 2018, Workshop
\confinvited {
\textbf{B. Bocklund*}, R.A. Otis*, Z.-K. Liu
(2018, May).
\emph{Software Workshop:} PyCalphad and ESPEI.
CALPHAD XLVII,
Juriquilla, Mexico.
}

% CALPHAD 2018
\confpresentation{
\textbf{B. Bocklund*}, A. Egorov, A. Obaied, R.A. Otis, I. Roslayakova, Z.-K. Liu
(2018, May).
ESPEI for Efficient Database Development, Modification and Uncertainty Quantification.
CALPHAD XLVII,
Juriquilla, Mexico.
}

% TMS 2018, Workshop
\confinvited {
\textbf{B. Bocklund}, R.A. Otis*, Z.-K. Liu
(2018, March).
\emph{Software Workshop:} PyCalphad and ESPEI.
TMS 2018 Annual Meeting,
Phoenix, AZ.
}

% TMS 2018
\confpresentation{
\textbf{B. Bocklund*}, R.A. Otis, Z.-K. Liu
(2018, March).
Thermodynamic Modeling with Uncertainty Quantifiation and its Implicitaions for Additive Manufacturing.
TMS 2018 Annual Meeting,
Phoenix, AZ.
}

% ICME 2017
\confposter{
\textbf{B. Bocklund*}, R.A. Otis, J. Paz Soldan-Palma, Y. Wang, Z.-K. Liu
(2017, May).
Automating Thermodynamic Database Development with ESPEI.
4th World Congress on Integrated Computational Materials Engineering,
Ypsilanti, MI.
}

\ifthenelse{\boolean{ENUMERATE_PRESENTATIONS}}
{\end{etaremune}}
{\end{itemize*}}



\vspace{-0.6em}
\hspace{0.5em}{\small \emph{\textbf{*} presenter}}
\vspace{0.4em}

}{}

% TODO: add organized symposia somewhere?

\ifthenelse{\boolean{USE_TECHNICAL_SKILLS}}{
\hrule
\vspace{-0.6em}
\subsection*{Technical Skills}

\begin{indentsection}{\parindent}
\hyphenpenalty=1000
\begin{description*}
  \item [Software Developed:]
  pycalphad (\href{https://pycalphad.org}{pycalphad.org}),
  ESPEI (\href{https://espei.org}{espei.org}),
  atomate (\href{https://atomate.org}{atomate.org})
  \item[Computational Tools and Software:]
  Python, MATLAB, VASP, Thermo-Calc, MongoDB
\end{description*}
\end{indentsection}
}
{} % Not technical skills blank

\end{document}
