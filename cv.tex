% resume.tex

\documentclass[10pt,letterpaper]{article}
\usepackage[letterpaper,margin=0.55in, top = 0.3in, bottom = 0.3in]{geometry}
\usepackage[utf8]{inputenc}
\usepackage{mdwlist}
\usepackage[T1]{fontenc}
\usepackage{textcomp}
\usepackage{amsmath}
\usepackage{tgpagella}
\usepackage[colorlinks=true, urlcolor=blue]{hyperref}
\pagestyle{empty}
\setlength{\tabcolsep}{0em}

% indent section style, used for sections that aren't already in lists
% that need indentation to the level of all text in the document
\newenvironment{indentsection}[1]%
{\begin{list}{}%
  {\setlength{\leftmargin}{#1}}%
  \item[]%
}
{\end{list}}

% opposite of above; bump a section back toward the left margin
\newenvironment{unindentsection}[1]%
{\begin{list}{}%
  {\setlength{\leftmargin}{-0.5#1}}%
  \item[]%
}
{\end{list}}

% format two pieces of text, one left aligned and one right aligned
\newcommand{\headerrow}[2]
%{\begin{tabular*}{\linewidth}{l@{\extracolsep{\fill}}r}
{\begin{tabular*}{\linewidth}{l@{\extracolsep{\fill}}r}
  #1 &
  #2 \\
\end{tabular*}}

% make "C++" look pretty when used in text by touching up the plus signs
\newcommand{\CPP}
{C\nolinebreak[4]\hspace{-.05em}\raisebox{.22ex}{\footnotesize\bf ++}}

% and the actual content starts here
\begin{document}
\begin{center}
{\LARGE \textbf{Brandon Bocklund}}

%ADDRESS LINE 1\ \ \textbullet
%\ \ ADDRESS LINE 2\ \ \textbullet
%\ \ CITY, ST #ZIP#
%\\


(269) 589-8602\ \ \textbullet
\ \ \href{mailto:bocklund@psu.edu}{bocklund@psu.edu}
\end{center}

\hrule
\vspace{-0.6em}
\subsection*{Research Experience}

\renewcommand\labelitemiii{$\circ$}
\begin{itemize}
  \parskip=0.1em
  \hyphenpenalty=1000
  
  \item
  \headerrow
    {\textbf{Phases Research Lab, Pennsylvania State University}}
    {\textbf{University Park, PA}}
  \\
  \headerrow
    {\emph{NSF Research Trainee (Advisor: Zi-Kui Liu)}}
    {\emph{2016 -- Present}}
  \begin{itemize*}
    \item Developer of pycalphad and ESPEI, open research and education software for computational thermodynamics
    \item Developer of atomate, a computational tool for high-throughput, first-principles DFT calculations with VASP
    \item Mentor undergraduate students in the Women In Science and Engineering
      Research (WISER) program
  \end{itemize*}

  \item
  \headerrow
    {\textbf{NASA Jet Propulsion Lab}}
    {\textbf{La Ca\~nada Flintridge, CA}}
  \\
  \headerrow
    {\emph{Graduate Research Intern (Mentors: Richard Otis, Peter Dillon)}}
    {\emph{05/2017 -- 08/2017}}
  \begin{itemize*}
    \item Used computational thermodynamics to develop bulk metallic glass alloy composition specifications
    \item Developed a model for oxygen tolerance in bulk metallic glasses 
  \end{itemize*}

  \item
  \headerrow
    {\textbf{Solid State Ionics Laboratory, Michigan State University}}
    {\textbf{East Lansing, MI}}
  \\
  \headerrow
    {\emph{Undergraduate Research Assistant (Advisor: Jason D. Nicholas)}}
    {\emph{2015 -- 2016}}
  \begin{itemize*}
    \item Fabricated and improved the performance of solid oxide fuel cells
    \item Characterized fuel cells with EIS, XRD, and SEM
    \item Developed Rp Plotter, a GUI-based Python application for data analysis and visualization
    \item Participated in a 10 week professional development course
  \end{itemize*}

  \item
  \headerrow
    {\textbf{Composite Materials \& Structures Center, Michigan State University}}
    {\textbf{East Lansing, MI}}
  \\
  \headerrow
    {\emph{Undergraduate Research Assistant (Advisor: Lawrence T. Drzal)}}
    {\emph{2014 -- 2015}}
  \begin{itemize*}
    \item Designed a graphene nanoplatlet-based capacitive deionization cell 
    \item Characterized graphene nanoplatelet papers using scanning electron microscopy
    \item Used Solidworks to create a 3D printed model for the deionization cell apparatus
    \item Participated in a 10 week professional development course
  \end{itemize*}
    
\end{itemize}

\hrule
\vspace{-0.6em}
\subsection*{Teaching Experience}

\renewcommand\labelitemiii{$\circ$}
\begin{itemize}
    \parskip=0.1em
    
    \item
    \headerrow
    {\textbf{Department of Materials Science and Engineering, Pennsylvania State University}}
    {\textbf{University Park, PA}}
    \\
    \headerrow
    {\emph{Teaching Assistant}}
    {\emph{2017}}
    \begin{itemize*}
        \item MatSE 462: General Properties Laboratory in Materials
            \begin{itemize}
                \item Independently taught and graded assignments for two lab sections of 10 students
                \item Instructed students on using techniques for characterizing mechanical, electrical and optical properties
            \end{itemize}
    \end{itemize*}
    
    \item
    \headerrow
    {\textbf{College of Engineering, Michigan State University}}
    {\textbf{East Lansing, MI}}
    \\
    \headerrow
    {\emph{Undergraduate Lab Mentor}}
    {\emph{2015 -- 2016}}
    \begin{itemize*}
        \item Mentored 3 classes, interacting with over 250 students
        \item Responsible for grading assignments and quizzes, promoting learning, and proctoring exams
        \begin{itemize}
            \item EGR 100: Introduction to Engineering Design
            \item EGR 102: Introduction to Engineering Modeling
            \item EGR 291: Spatial Visualization
        \end{itemize}
    \end{itemize*}
\end{itemize}


\hrule
\vspace{-0.6em}
\subsection*{Education}

\begin{itemize}
  \parskip=0.1em

  \item
  \headerrow
    {\textbf{Pennsylvania State University}}
    {\textbf{University Park, PA}}
  \\
  \headerrow
    {\emph{Ph.D. Materials Science and Engineering; Graduate Minor, Computational Materials}}
    {\emph{2016 -- Present}}
  \begin{itemize*}
    \item 3.86 GPA
    \item NSF Research Trainee in the CoMET Program (\href{http://dftcomet.psu.edu}{dftcomet.psu.edu})
    \item Helen R. and Van H. Leichliter Graduate Fellowship recipient (2016)
  \end{itemize*}


  \item 
  \headerrow
    {\textbf{Michigan State University}}
    {\textbf{East Lansing, MI}}
  \\
  \headerrow
    {\emph{B.S. Materials Science and Engineering}}
    {\emph{2012 -- 2016}}
  \begin{itemize*}
    \item 3.56 GPA
    \item Dean's List, 5 semesters
    \item MSU College of Engineering Endowed Opportunity Fund scholarship recipient (2015 -- 2016)
  \end{itemize*}

\end{itemize}

\hrule
\vspace{-0.6em}
\subsection*{Publications}

\begin{itemize*}
\item Mathew, K., Montoya, J.H., Faghaninia, A., Dwarakanath, S., Aykol, M.,
  Tang, H., Chu, I., Smidt, T., \textbf{Bocklund, B.}, Horton, M., Dagdelen, J., Wood, B., Liu, Z.-K., Neaton, J., Ong, S.P., Persson, K., Jain, A.
 Atomate: A high-level interface to generate, execute,
  and analyze computational materials science workflows. \textbf{Computational Materials Science 139, 140–152 (2017).}
  \href{https://doi.org/10.1016/j.commatsci.2017.07.030}{doi: 10.1016/j.commatsci.2017.07.030}

\item Bobbio, L.D., \textbf{Bocklund, B.}, Otis, R, Borgonia, J.P.,
  Dillon, R.P., Shapiro, A.A., McEnerney, B., Liu, Z.-K., Beese, A.M.
  Characterization of a functionally graded material of Ti-6Al-4V to 304L stainless steel with an intermediate V section. \emph{Submitted.}

\item Smith, N. D., Marker, C., \textbf{Bocklund, B.}, Orabona, N., Lichtenstein, T.,
  Gesualdi, J., Nigl, T., Liu, Z.-K., Kim, H. Thermodynamic Assessment of the
  Strontium-Antimony Binary System via Electromotive Force Measurements
  Supported by First-Principles Calculations and CALPHAD Modeling. \emph{Submitted.}

\item Bobbio, L.D., \textbf{Bocklund, B.}, Otis, R., Borgonia, J.P., Dillon, R.P., Shapiro,
  A.A., McEnerney, Bryan, Liu, Z.-K., Beese, A.M.
  Experimental analysis and thermodynamic calculations of an additively manufactured functionally graded material of V to Invar 36. \emph{In preparation.}
  
\item Marker, C., Higgins, O., Smith, N., \textbf{Bocklund, B.}, Kim, H., Liu, Z.-K.
   Thermodynamic modeling of the Ba-Sb system from first-principles
   calculations. \emph{In preparation.}

\item \textbf{Bocklund, B.}, Otis, R. A., Egorov, A., Roslyakova, I., Liu, Z.-K.
  ESPEI for efficient database development, modification and uncertainty
  quantification: application to the Cu-Mg system. \emph{In preparation.}
 
\end{itemize*}

\hrule
\vspace{-0.6em}
\subsection*{Technical Skills}

\begin{indentsection}{\parindent}
\hyphenpenalty=1000
\begin{description*}
  \item [Software Developed:]
  pycalphad (\href{http://pycalphad.org}{pycalphad.org}),
  ESPEI (\href{http://espei.org}{espei.org}),
  atomate (\href{http://hackingmaterials.github.io/atomate}{hackingmaterials.github.io/atomate})
  \item[Computational Tools and Software:]
  Python, MATLAB, VASP, Thermo-Calc, MongoDB
\end{description*}
\end{indentsection}
\end{document}
