% resume.tex

\documentclass[10pt,letterpaper]{article}
\usepackage[letterpaper,margin=0.55in, top = 0.3in, bottom = 0.3in]{geometry}
\usepackage[utf8]{inputenc}
\usepackage{mdwlist}
\usepackage[T1]{fontenc}
\usepackage{textcomp}
\usepackage{amsmath}
\usepackage{tgpagella}
\usepackage[colorlinks=true, urlcolor=blue]{hyperref}
\pagestyle{empty}
\setlength{\tabcolsep}{0em}

\usepackage{fontspec}
\defaultfontfeatures{Path={/Users/brandon/Adobe-Utopia-fonts-named/}}
\setmainfont[
  UprightFont=*-Regular,
  ItalicFont=*-Italic,
  BoldFont=*-Semibold,
  BoldItalicFont=*-SemiboldIt,
]{UtopiaStd}


% indent section style, used for sections that aren't already in lists
% that need indentation to the level of all text in the document
\newenvironment{indentsection}[1]%
{\begin{list}{}%
  {\setlength{\leftmargin}{#1}}%
  \item[]%
}
{\end{list}}

% opposite of above; bump a section back toward the left margin
\newenvironment{unindentsection}[1]%
{\begin{list}{}%
  {\setlength{\leftmargin}{-0.5#1}}%
  \item[]%
}
{\end{list}}

% format two pieces of text, one left aligned and one right aligned
\newcommand{\headerrow}[2]
%{\begin{tabular*}{\linewidth}{l@{\extracolsep{\fill}}r}
{\begin{tabular*}{\linewidth}{l@{\extracolsep{\fill}}r}
  #1 &
  #2 \\
\end{tabular*}}

% make "C++" look pretty when used in text by touching up the plus signs
\newcommand{\CPP}
{C\nolinebreak[4]\hspace{-.05em}\raisebox{.22ex}{\footnotesize\bf ++}}

% and the actual content starts here
\begin{document}
\begin{center}
{\LARGE \textbf{Brandon Bocklund}}

%ADDRESS LINE 1\ \ \textbullet
%\ \ ADDRESS LINE 2\ \ \textbullet
%\ \ CITY, ST #ZIP#
%\\


(269) 589-8602\ \ \textbullet
\ \ \href{mailto:bocklund@psu.edu}{bocklund@psu.edu}
\end{center}

\hrule
\vspace{-0.6em}
\subsection*{Research Experience}

\renewcommand\labelitemiii{$\circ$}
\begin{itemize}
  \parskip=0.1em
  \hyphenpenalty=1000

  \item
  \headerrow
    {\textbf{Phases Research Lab, Pennsylvania State University}}
    {\textbf{University Park, PA}}
  \\
  \headerrow
    {\emph{NASA Space Technology Research Fellow (Advisor: Zi-Kui Liu)}}
    {\emph{2016 -- Present}}
  \begin{itemize*}
    \item Developer of pycalphad and ESPEI, open research and education software for computational thermodynamics
    \item Developer of atomate, a computational tool for high-throughput, first-principles DFT calculations with VASP
    \item Mentor undergraduate students in the Women In Science and Engineering
      Research (WISER) program
  \end{itemize*}

  \item
  \headerrow
    {\textbf{NASA Jet Propulsion Lab}}
    {\textbf{La Ca\~nada Flintridge, CA}}
  \\
  \headerrow
    {\emph{Graduate Research Intern (Mentors: Richard Otis, Peter Dillon)}}
    {\emph{05/2017 -- 08/2017}}
  \begin{itemize*}
    \item Used computational thermodynamics to develop bulk metallic glass alloy composition specifications
    \item Developed a model for oxygen tolerance in bulk metallic glasses
  \end{itemize*}

  \item
  \headerrow
    {\textbf{Solid State Ionics Laboratory, Michigan State University}}
    {\textbf{East Lansing, MI}}
  \\
  \headerrow
    {\emph{Undergraduate Research Assistant (Advisor: Jason D. Nicholas)}}
    {\emph{2015 -- 2016}}
  \begin{itemize*}
    \item Fabricated and improved the performance of solid oxide fuel cells
    \item Characterized fuel cells with EIS, XRD, and SEM
    \item Developed Rp Plotter, a GUI-based Python application for data analysis and visualization
    \item Participated in a 10 week professional development course
  \end{itemize*}

  \item
  \headerrow
    {\textbf{Composite Materials \& Structures Center, Michigan State University}}
    {\textbf{East Lansing, MI}}
  \\
  \headerrow
    {\emph{Undergraduate Research Assistant (Advisor: Lawrence T. Drzal)}}
    {\emph{2014 -- 2015}}
  \begin{itemize*}
    \item Designed a graphene nanoplatlet-based capacitive deionization cell
    \item Characterized graphene nanoplatelet papers using scanning electron microscopy
    \item Used Solidworks to create a 3D printed model for the deionization cell apparatus
    \item Participated in a 10 week professional development course
  \end{itemize*}

\end{itemize}

\hrule
\vspace{-0.6em}
\subsection*{Teaching Experience}

\renewcommand\labelitemiii{$\circ$}
\begin{itemize}
    \parskip=0.1em

    \item
    \headerrow
    {\textbf{Department of Materials Science and Engineering, Pennsylvania State University}}
    {\textbf{University Park, PA}}
    \\
    \headerrow
    {\emph{Teaching Assistant}}
    {\emph{2017}}
    \begin{itemize*}
        \item MatSE 404/BME 444: Surfaces and the Biological Response to Materials
            \begin{itemize}
                \item Developed and graded problems for homework and exams
            \end{itemize}
        \item MatSE 462: General Properties Laboratory in Materials
            \begin{itemize}
                \item Independently taught and graded assignments for two lab sections of 10 students
                \item Instructed students on using techniques for characterizing mechanical, electrical and optical properties
            \end{itemize}
    \end{itemize*}

    \item
    \headerrow
    {\textbf{College of Engineering, Michigan State University}}
    {\textbf{East Lansing, MI}}
    \\
    \headerrow
    {\emph{Undergraduate Lab Mentor}}
    {\emph{2015 -- 2016}}
    \begin{itemize*}
        \item Mentored 3 classes, interacting with over 250 students
        \item Responsible for grading assignments and quizzes, promoting learning, and proctoring exams
        \begin{itemize}
            \item EGR 100: Introduction to Engineering Design
            \item EGR 102: Introduction to Engineering Modeling
            \item EGR 291: Spatial Visualization
        \end{itemize}
    \end{itemize*}
\end{itemize}


\hrule
\vspace{-0.6em}
\subsection*{Education}

\begin{itemize}
  \parskip=0.1em

  \item
  \headerrow
    {\textbf{Pennsylvania State University}}
    {\textbf{University Park, PA}}
  \\
  \headerrow
    {\emph{Ph.D., Materials Science and Engineering; Graduate Minor, Computational Materials}}
    {\emph{2016 -- Present}}
  \begin{itemize*}
    \item 3.73 GPA
    \item NASA Space Technology Research Fellow (2018 -- Present)
    \item NSF Research Trainee in the CoMET Program (\href{http://dftcomet.psu.edu}{dftcomet.psu.edu}) (2016 -- 2018)
  \end{itemize*}


  \item
  \headerrow
    {\textbf{Michigan State University}}
    {\textbf{East Lansing, MI}}
  \\
  \headerrow
    {\emph{B.S. Materials Science and Engineering}}
    {\emph{2012 -- 2016}}
  \begin{itemize*}
    \item 3.56 GPA
    \item Dean's List, 5 semesters
  \end{itemize*}

\end{itemize}

\hrule
\vspace{-0.6em}
\subsection*{Awards and Honors}

\begin{itemize}
  \parskip=0.1em

  \item
  \headerrow
       {pycalphad: Runner Up, NASA Software of the Year (SoY) award}
       {\emph{2019}}
  \item
  \headerrow
       {Honorable Mention, National Science Foundation Graduate Research Fellowship Program}
       {\emph{2018}}
  \item
  \headerrow
    {Outstanding Contribution in Reviewing - \emph{CALPHAD}}
    {\emph{2017}}
  \item
  \headerrow
    {Helen R. and Van H. Leichliter Graduate Fellowship recipient} {\emph{2016}}
  \item
  \headerrow
    {MSU College of Engineering Endowed Opportunity Fund scholarship recipient} {\emph{2015}}

\end{itemize}


\hrule
\vspace{-0.6em}
\subsection*{Publications}

\begin{itemize*}

% Formatting:
% \item <Authors>
% <Title>
% <Publication>
% <Link>

%%%%%%%%%%%% Submitted %%%%%%%%%%%%
\item A. Obaied, \textbf{B. Bocklund}, S. Zomorodpoosh, L. Zhang, R. Otis, Z.-K. Liu, I. Roslyakova,
  Thermodynamic re-assessment of pure chromium using modified segmented regression model,
  CALPHAD \textbf{Accepted} (2019).

%%%%%%%%%%%% 2020 %%%%%%%%%%%%
% FGMs: Sigma in SS304L compared to SS420
\item L.D. Bobbio, \textbf{B. Bocklund}, A. Reichardt, R.A. Otis, J.P. Borgonia, R.P. Dillon, A.A. Shapiro, B.W. McEnerney, P. Hosemann, Z.-K. Liu, A.M. Beese,
Analysis of formation and growth of the $ \sigma $ phase in additively manufactured functionally graded materials,
\textbf{Journal of Alloys and Compounds} 814 (2020) 151729.
\href{https://doi.org/10.1016/j.jallcom.2019.151729}{doi:10.1016/j.jallcom.2019.151729}.

%%%%%%%%%%%% 2019 %%%%%%%%%%%%
% ESPEI
\item \textbf{B. Bocklund}, R.A. Otis, A. Egorov, A. Obaied, I. Roslyakova, Z.-K. Liu,
  ESPEI for efficient thermodynamic database development, modification, and uncertainty quantification: application to Cu-Mg,
  \textbf{MRS Communications} 9(2) (2019) 618-627.
  \href{https://doi.org/10.1557/mrc.2019.59}{doi:10.1557/mrc.2019.59}.

% Noah phase diagram UQ
\item N.H. Paulson, \textbf{B. Bocklund}, R.A. Otis, Z.-K. Liu, S. Marius,
  Quantified Uncertainty in Thermodynamic Modeling for Materials Design.
  \textbf{Acta Materialia} 174 (2019) 9-15.
  \href{https://doi.org/10.1016/j.actamat.2019.05.017}{doi:10.1016/j.actamat.2019.05.017}.

%%%%%%%%%%%% 2018 %%%%%%%%%%%%
% Seebeck
\item Y. Wang, Y.-J. Hu, \textbf{B. Bocklund}, S.-L. Shang, B.-C. Zhou, Z.-K. Liu, L.-Q. Chen,
  First-principles thermodynamic theory of Seebeck coefficients,
  \textbf{Physical Review B} 98 (2018) 224101.
  \href{https://doi.org/10.1103/PhysRevB.98.224101}{doi:10.1103/PhysRevB.98.224101}.

% FGMs: Ti-6Al-4V to 304L
\item L.D. Bobbio, \textbf{B. Bocklund}, R.A. Otis, J.P. Borgonia, R.P. Dillon, A.A. Shapiro, B. McEnerney, Z.-K. Liu, A.M. Beese,
  Characterization of a functionally graded material of Ti-6Al-4V to 304L stainless steel with an intermediate V section. \textbf{Journal of Alloys and Compounds} 742 (2018) 1031-1036.
  \href{https://doi.org/10.1016/j.jallcom.2018.01.156}{doi: 10.1016/j.jallcom.2018.01.156}

% FGMs: V to Invar36
\item L.D. Bobbio, \textbf{B. Bocklund}, R.A. Otis, J.P. Borgonia, R.P. Dillon, A.A. Shapiro, B. McEnerney, Z.-K. Liu, A.M. Beese,
  Experimental analysis and thermodynamic calculations of an additively manufactured functionally graded material of V to Invar 36,
  \textbf{Journal of Materials Research} 33 (2018) 1642–1649.
  \href{https://doi.org/10.1016/10.1557/jmr.2018.92}{doi:10.1557/jmr.2018.92}.

%%%%%%%%%%%% 2017 %%%%%%%%%%%%
% atomate
\item K. Mathew, J.H. Montoya, A. Faghaninia, S. Dwarakanath, M. Aykol,
  H. Tang, I. Chu, T. Smidt, \textbf{B. Bocklund}, M. Horton, J. Dagdelen, B. Wood, Z.-K. Liu, J. Neaton, S.P. Ong, K. Persson, A. Jain,
 Atomate: A high-level interface to generate, execute,
  and analyze computational materials science workflows. \textbf{Computational Materials Science} 139, 140–152 (2017).
  \href{https://doi.org/10.1016/j.commatsci.2017.07.030}{doi: 10.1016/j.commatsci.2017.07.030}

\end{itemize*}

\hrule
\vspace{-0.6em}
\subsection*{Technical Skills}

\begin{indentsection}{\parindent}
\hyphenpenalty=1000
\begin{description*}
  \item [Software Developed:]
  pycalphad (\href{https://pycalphad.org}{pycalphad.org}),
  ESPEI (\href{https://espei.org}{espei.org}),
  atomate (\href{https://atomate.org}{atomate.org})
  \item[Computational Tools and Software:]
  Python, MATLAB, VASP, Thermo-Calc, MongoDB
\end{description*}
\end{indentsection}
\end{document}
